\usepackage{graphicx}
\usepackage{booktabs}
\usepackage{amsmath}
\usepackage{amssymb}
\usepackage{amsthm}
\usepackage{hyperref}
\usepackage{tikz}
\usepackage{cite}
\usepackage{url}

% font configuration
\usepackage{fontspec}
\setmainfont[Mapping=tex-text]{Old Standard}
\setromanfont[Mapping=tex-text]{Old Standard}
\setsansfont[Mapping=tex-text]{Open Sans}
\setmonofont[Scale=0.87]{Fira Mono}
\newfontfamily\cyrillicfont[Mapping=tex-text]{Old Standard}
\newfontfamily\cyrillicfontrm[Mapping=tex-text]{Old Standard}
\newfontfamily\cyrillicfontsf[Mapping=tex-text]{Open Sans}
\newfontfamily\cyrillicfonttt[Scale=0.87]{Fira Mono}
\newfontfamily\rmfamily[Mapping=tex-text]{Old Standard}
\newfontfamily\sffamily[Mapping=tex-text]{Open Sans}
\renewcommand{\familydefault}{\rmdefault}

% language configuration
\usepackage{polyglossia}
\setdefaultlanguage{russian}
\setotherlanguage{english}

\usepackage{latexsym} % \Box macro
\usepackage{mathtools} % fancy dots in matrices

\raggedbottom
\clubpenalty=10000
\widowpenalty=10000

% automatically find figures from build directory
\graphicspath{{build/}}

% multiple bibliographies
\usepackage[resetlabels]{multibib}

% custom mathematical expressions
\newcommand{\Var}[1]{\sigma_{#1}^2}
\newcommand{\Fourier}[1]{\mathcal{F}\left\{#1\right\}}
\newcommand{\InverseFourier}[1]{\mathcal{F}^{-1}\left\{#1\right\}}
\newcommand{\Fun}[1]{\mathcal{D}_1\left(x,#1\right)}
\newcommand{\FunSecond}[1]{\mathcal{D}_2\left(x,#1\right)}
\newcommand{\FunThird}[1]{\mathcal{D}_2\left(x,#1\right)}
\newcommand{\Sinh}[1]{\cosh\left(#1\right)}
\newcommand{\SinhX}[1]{\sinh\left(#1\right)}

\newcommand{\FourierY}[2]{\mathcal{F}_{#2}\!\left\{#1\right\}}
\newcommand{\InverseFourierY}[2]{\mathcal{F}^{-1}_{#2}\!\left\{#1\right\}}

\newcommand{\FourierX}[3]{\mathcal{F}_{#2}\!\left\{#1\right\}\!\left(#3\right)}
\newcommand{\InverseFourierX}[3]{\mathcal{F}^{-1}_{#2}\!\left\{#1\right\}\!\left(#3\right)}

% properly aligned version of sqrt for \zeta_y^2
\newcommand{\SqrtZeta}[1]{\sqrt{\vphantom{\zeta_x^2}\smash[b]{#1}}}

% wave vector
\newcommand{\Kvec}{\vec{k}}
\newcommand{\Kveclen}{\lvert\smash[b]{\Kvec}\rvert}
